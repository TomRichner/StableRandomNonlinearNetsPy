```latex
% In your preamble:
% \usepackage{booktabs,tabularx,siunitx,threeparttable}
% \sisetup{detect-all}

\begin{table}[htbp]
\centering
\begin{threeparttable}
\caption{State variables and parameters of the RNN with spike-frequency adaptation (SFA) and short-term depression (STD).}
\label{tab:rnn-sfa-std}
\setlength{\tabcolsep}{6pt}
\renewcommand{\arraystretch}{1.15}
\begin{tabularx}{\linewidth}{@{} l l l l X @{}}
\toprule
\textbf{Symbol} & \textbf{Description} & \textbf{Value} & \textbf{Dims.} & \textbf{Notes} \\
\midrule
$n_E$ & \# excitatory neurons & $7$ & $1$ & $70\%$ excitatory neurons (of $n_E+n_I=10$). \\
$n_I$ & \# inhibitory neurons & $3$ & $1$ & $30\%$ inhibitory neurons. \\
$r$ & spike rate & dependent var. & $10\times1$ & Rectified rates, $r\ge 0$; \si{\hertz}. \\
$u_d$ & dendrite & state var. & $10\times1$ & \si{\hertz}. \\
$\tau_d$ & dendritic time const. & $25$ & $1$ & \si{\milli\second}. \\
$a_k$ & SFA & state var. & $10\times3$ & Three SFA variables per neuron; \si{\hertz}. \\
$c_{\mathrm{SFA}}$ & SFA strength & $0$ or $1/6$ & $10\times1$ & $0$ for inhibitory, $1/6$ for excitatory; unitless. With three SFA vars per excitatory neuron, total SFA strength is $1/2$. \\
$\tau_{a_k}$ & SFA time constants & $[0.3,\,2.12,\,15]$ & $3\times1$ per neuron & Log-spaced; \si{\second}. \\
$u_{\mathrm{ex}}$ & external input & user defined & $10\times1$ & \si{\hertz}. \\
$b$ & STD & state var. & $10\times1$ & One STD variable applied to each excitatory neuron's output; unitless. For excitatory neurons $0\le b\le 1$; for inhibitory neurons $b=1$. \\
$\tau_{\mathrm{STD}}$ & STD onset time const. & $1/2$ & $1$ & \si{\second}. \\
$\tau_b$ & STD recovery time const. & $2$ & $1$ & \si{\second}. \\
$M$ & connection matrix & parameterized & $10\times10$ & Sparse with Gaussian random weights; no self-connections; unitless. \\
$\mathbb{E}[w_{EE}]$ & mean E$\to$E weight & $0.5$ & \textemdash & $0\le w_{EE}$; unitless. \\
$\mathbb{E}[w_{EI}]$ & mean E$\to$I weight & $0.5$ & \textemdash & $0\le w_{EI}$. \\
$\mathbb{E}[w_{IE}]$ & mean I$\to$E weight & $0.5$ & \textemdash & $w_{IE}\le 0$. \\
$\mathbb{E}[w_{II}]$ & mean I$\to$I weight & $0.25$ & \textemdash & $w_{II}\le 0$. \\
\bottomrule
\end{tabularx}
\end{threeparttable}
\end{table}
```